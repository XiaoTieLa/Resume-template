%% 
%% Copyright (c) 2018-2020 Weitian LI <wt@liwt.net> 
%% CC BY 4.0 License 
%% 
%% Created: 2018-04-11 
%% 

% Chinese version 

\documentclass[zh]{resume} 

% File information shown at the footer of the last page 
\fileinfo{%
  \faCopyright{} 2024, 李明 \hspace{0.5em} 
  \creativecommons{by}{4.0} \hspace{0.5em} 
  \githublink{liming}{resume} \hspace{0.5em} 
  \faEdit{} \today 
} 

\name{明}{李} 

\keywords{Python, 机器学习, 数据分析, 并行计算, JavaScript, 人工智能} 

\profile{ 
  \mobile{(+86) 138-1234-5678} 
  \email{example@qq.com} 
  \github{ExampleCoder} \\ 
  \university{清华大学} 
  \degree{软件工程 \textbullet 硕士} 
  \address{北京} 
} 

\begin{document} 
\makeheader 

\vspace{-1.3em} 
%====================================================================== 
\sectionTitle{教育背景}{\faGraduationCap} 
%====================================================================== 
\begin{educations} 
  \education% 
    {2025.06}% 
    [2021.09]% 
    {清华大学}% 
    {计算机科学与技术系}% 
    {软件工程}% 
    {硕士} 
    
  \separator{0.7ex} 
  \education% 
    {}% 
    [2023.07]% 
    {校级荣誉}% 
    {校优秀毕业生}% 
    {校一等奖学金}% 
    {综测专业第一} 
  
\end{educations} 

\vspace{-1.3em} 

%====================================================================== 
\sectionTitle{技能和语言}{\faWrench} 
%====================================================================== 
\begin{competences} 
  \comptence{编程语言}{% 
    Python(主要),熟悉异步编程和装饰器,对数据科学库(如Pandas、NumPy)有深入了解;JavaScript 
  } 
  \comptence{机器学习}{% 
    熟悉TensorFlow和PyTorch框架,了解深度学习模型的训练与优化,能够进行模型部署和推理 
  } 
  \comptence{数据分析}{% 
    熟练使用数据可视化工具(如Matplotlib、Seaborn),能够进行数据清洗、特征工程和统计分析 
  } 
  \comptence{并行计算}{% 
    熟悉MPI和OpenMP并行编程模型,能够设计高效的并行算法,优化计算性能 
  } 
  \comptence{云计算}{% 
    熟悉AWS和Azure云服务平台,能够进行云资源的管理和部署 
  } 
\end{competences} 

\vspace{-1.3em} 
%====================================================================== 
\sectionTitle{竞赛获奖}{\faAtom} 
%====================================================================== 
\begin{itemize} 
  \item 全国大学生数学建模竞赛 \textbf{一等奖} \hfill \textcolor{symbolcolor}{2023.09} 
  \item 中国研究生数学建模竞赛 \textbf{二等奖} \hfill \textcolor{symbolcolor}{2024.05} 
  \item 国际大学生程序设计竞赛(区域赛)\textbf{银奖} \hfill \textcolor{symbolcolor}{2024.10} 
  \item 全国大学生机器人大赛 \textbf{一等奖} \hfill \textcolor{symbolcolor}{2024.07} 
\end{itemize} 

\vspace{-0.4em} 
%====================================================================== 
\sectionTitle{实习经历}{\faBriefcase} 
%====================================================================== 

% 定义实习条目样式(带标识系统) 
\newcommand{\experienceitem}[4]{% 
  \noindent 
  \begin{minipage}[t]{\linewidth} 
    % 第一层级:公司名称 
    {\Large\bfseries\textcolor{accentcolor}{#1}}\par  % 调整为Large字号 
    
    % 第二层级:职位+日期(带标识图标) 
    \vspace{0.3em} 
    \hspace*{0.7em}  % 图标缩进 
    \icon{\faUserCog}  % 岗位标识图标 
    {\large\textcolor{linkcolor}{#2}}  % 岗位名称 
    \hfill  % 右对齐日期 
    {\normalsize\bfseries\textcolor{gray!60!black}{\faCalendar*~#3}}\par  % 日期带日历图标 
    
    \vspace{0.8em} 
    % 内容区块 
    #4 
  \end{minipage} 
  \vspace{1.2em} 
} 

\begin{itemize}[leftmargin=1.0em, nosep, before=\vspace{-0.5em}] 
  \vspace{0.6em} 
  \item[] 
  \experienceitem{百度——人工智能实验室} % 公司名称 
    {机器学习工程师实习生} % 职位名称 
    {2024.07 - 2024.12} % 日期 
    {\begin{itemize}[label=\faAngleRight, leftmargin=2.2em, nosep, topsep=0pt] 
       \item 参与深度学习模型的优化工作,使用PyTorch框架对图像分类模型进行训练和调优,提升了模型的准确率和运行效率 
       \item 负责数据预处理和特征提取,使用Python编写数据清洗脚本,优化了数据处理流程,提高了数据质量 
       \item 参与了模型的部署工作,将训练好的模型部署到服务器上,提供了RESTful API接口,支持在线预测 
     \end{itemize}} 
    
  \vspace{-1em} 
  \item[] 
  \experienceitem{阿里巴巴——数据科学与技术研究院} % 公司名称 
    {数据分析师实习生} % 职位名称 
    {2024.03 - 2024.06} % 日期 
    {\begin{itemize}[label=\faAngleRight, leftmargin=2.2em, nosep, topsep=0pt] 
       \item 负责大数据分析项目,使用Hadoop和Spark进行数据处理和分析,优化了数据处理流程,提高了分析效率 
       \item 参与了数据可视化工作,使用Tableau和Power BI工具进行数据可视化,提供了直观的分析报告 
       \item 负责数据清洗和特征工程,使用Python和SQL进行数据处理,提高了数据质量 
     \end{itemize}} 
\end{itemize} 

\vspace{-1.5em} 
%====================================================================== 
\sectionTitle{项目经历}{\faCode} 
%====================================================================== 

% 定义项目条目样式 
\newcommand{\projectitem}[3]{% 
  \vspace{0.2em} 
  \noindent 
  \begin{minipage}[t]{\linewidth} 
    {\large\bfseries\textcolor{accentcolor}{#1}} \hfill {\small\textcolor{gray}{#2}}\par 
    \vspace{0.3em} 
    #3 
  \end{minipage} 
  \vspace{0.8em} 
} 

% 项目列表 
\begin{itemize}[leftmargin=1.5em, nosep, before=\vspace{-0.5em}, after=\vspace{-0.5em}] 
  \item[] 
  \projectitem{基于深度学习的图像识别系统} 
    {技术栈:Python, TensorFlow, Keras, OpenCV} 
    {\begin{itemize}[label=\faAngleRight, leftmargin=2.2em, nosep, topsep=0pt] 
       \item \textbf{模型设计}:使用卷积神经网络(CNN)构建图像分类模型,支持多种图像格式的输入 
       \item \textbf{数据增强}:实现了数据增强模块,包括旋转、翻转、裁剪等操作,提升了模型的泛化能力 
       \item \textbf{模型部署}:将训练好的模型部署到服务器上,提供了RESTful API接口,支持在线预测 
     \end{itemize}} 
     
  \vspace{-0.8em} 
  \item[] 
  \projectitem{分布式数据处理框架} 
    {技术栈:Python, MPI, OpenMP, NumPy} 
    {\begin{itemize}[label=\faAngleRight, leftmargin=2.2em, nosep, topsep=0pt] 
        \item \textbf{并行计算}:使用MPI和OpenMP实现并行计算,优化了数据处理流程,提升了计算效率 
        \item \textbf{任务调度}:设计了任务调度模块,支持动态任务分配和负载均衡 
        \item \textbf{性能优化}:通过优化内存管理和算法实现,减少了计算时间和资源消耗 
     \end{itemize}} 
    
  \vspace{-0.8em} 
  \item[] 
  \projectitem{基于深度学习的语音识别系统} 
    {技术栈:Python, TensorFlow, Keras, librosa} 
    {\begin{itemize}[label=\faAngleRight, leftmargin=2.2em, nosep, topsep=0pt] 
        \item \textbf{模型设计}:使用循环神经网络(RNN)构建语音识别模型,支持多种语音格式的输入 
        \item \textbf{数据预处理}:实现了语音数据的预处理模块,包括降噪、分段、特征提取等操作 
        \item \textbf{模型部署}:将训练好的模型部署到服务器上,提供了RESTful API接口,支持在线预测 
     \end{itemize}} 
\end{itemize} 
\end{document}